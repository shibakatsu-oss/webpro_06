\documentclass[a4paper,11pt]{jsarticle}
\usepackage[dvipdfmx]{graphicx}
\usepackage{url}
\usepackage{listings}
\usepackage{ascmac} 


\title{Webプログラミング 最終課題レポート}
\author{251134 横居 克憲} 
\date{\today}

\begin{document}

\maketitle

\begin{screen}
\textbf{ソースコード (GitHubリポジトリURL):}\\
\url{https://github.com/shibakatsu-oss/webpro_06}
\end{screen}

\vspace{1cm} 

\section{利用者向け仕様書}
% 3つの内の代表的なシステムについて1つ記述

\subsection{システム概要}
本システムは,うまい棒の情報を管理・閲覧するためのWebアプリケーションである.
利用者は,うまい棒の味,価格,カロリーなどを一覧で確認できるほか,詳細情報の閲覧や新規登録が可能である.

\subsection{操作方法}
\begin{enumerate}
    \item \textbf{一覧画面}: トップページにアクセスすると,現在登録されているうまい棒の一覧が表示される.
    \item \textbf{詳細表示}: 一覧の「味」のリンクをクリックすると,そのうまい棒の詳細情報(価格・カロリー)が表示される.
    \item \textbf{新規登録}: 一覧画面下部の「新規追加」ボタンを押すと,登録画面に遷移する.必要な情報を入力して「登録」を押すとデータが保存される.
    \item \textbf{削除}: 一覧画面または詳細画面にある「削除」リンクをクリックすると,該当データが削除される.
\end{enumerate}


\section{管理者向け仕様書}
% 管理者からすると1つのシステムのため全体で1つ記述

\subsection{システムのセットアップ}
本システムはNode.js上で動作する.以下の手順でサーバーを起動する.

\begin{verbatim}
$ npm install express ejs
$ node app5.js
\end{verbatim}

\subsection{動作環境}
\begin{itemize}
    \item Node.js 
    \item Webブラウザ
\end{itemize}


\section{開発者向け仕様書}
%  システムごとに1つで計3つ記述

\subsection{システム1: うまい棒管理システム}

\subsubsection{データ構造}
データは配列 \texttt{umaiboData} 内のオブジェクトとして管理する.
各オブジェクトは以下のプロパティを持つ.

\begin{table}[h]
    \centering
    \begin{tabular}{|l|l|l|}
        \hline
        プロパティ名 & 型 & 説明 \\
        \hline
        id & 数値 & データ固有のID \\
        flavor & 文字列 & うまい棒の味(例: メンタイ) \\
        price & 数値 & 価格(円) \\
        calorie & 数値 & カロリー(kcal) \\
        \hline
    \end{tabular}
\end{table}

\subsubsection{画面遷移とリソース}
HTTPメソッドとリソース名(URL)の対応は以下の通りである.

\begin{itemize}
    \item \textbf{一覧表示} (GET \texttt{/umaibo}): 全データを一覧表示する.
    \item \textbf{詳細表示} (GET \texttt{/umaibo/:number}): 指定IDの詳細を表示する.
    \item \textbf{新規登録フォーム} (GET \texttt{/umaibo/create}): 登録画面へ遷移する.
    \item \textbf{新規登録処理} (POST \texttt{/umaibo}): データを追加し一覧へ戻る.
    \item \textbf{編集フォーム} (GET \texttt{/umaibo/edit/:number}): 編集画面を表示する.
    \item \textbf{更新処理} (POST \texttt{/umaibo/update/:number}): データを更新する.
    \item \textbf{削除処理} (GET \texttt{/umaibo/delete/:number}): データを削除する.
\end{itemize}

\subsection{システム2: スマブラキャラ一覧システム}

\subsubsection{データ構造}
データは配列 \texttt{smashData} 内のオブジェクトとして管理する.

\begin{table}[h]
    \centering
    \begin{tabular}{|l|l|l|}
        \hline
        プロパティ名 & 型 & 説明 \\
        \hline
        id & 数値 & データ固有のID \\
        name & 文字列 & ファイター名(例: マリオ) \\
        series & 文字列 & 出典作品(例: スーパーマリオ) \\
        number & 文字列 & 参戦ナンバー(例: 01) \\
        \hline
    \end{tabular}
\end{table}

\subsubsection{画面遷移とリソース}
\begin{itemize}
    \item \textbf{一覧表示} (GET \texttt{/smash}): 全キャラを一覧表示する.
    \item \textbf{詳細表示} (GET \texttt{/smash/:index}): 指定キャラの詳細を表示する.
    \item \textbf{新規登録フォーム} (GET \texttt{/smash/create}): 登録画面へ遷移する.
    \item \textbf{新規登録処理} (POST \texttt{/smash}): データを追加する.
    \item \textbf{編集フォーム} (GET \texttt{/smash/edit/:index}): 編集画面を表示する.
    \item \textbf{更新処理} (POST \texttt{/smash/update/:index}): データを更新する.
    \item \textbf{削除処理} (GET \texttt{/smash/delete/:index}): データを削除する.
\end{itemize}

\subsection{システム3: スプラトゥーン武器一覧システム}

\subsubsection{データ構造}
データは配列 \texttt{splaData} 内のオブジェクトとして管理する.

\begin{table}[h]
    \centering
    \begin{tabular}{|l|l|l|}
        \hline
        プロパティ名 & 型 & 説明 \\
        \hline
        id & 数値 & データ固有のID \\
        name & 文字列 & ブキ名(例: わかばシューター) \\
        sub & 文字列 & サブウェポン(例: スプラッシュボム) \\
        special & 文字列 & スペシャルウェポン(例: グレートバリア) \\
        \hline
    \end{tabular}
\end{table}

\subsubsection{画面遷移とリソース}
\begin{itemize}
    \item \textbf{一覧表示} (GET \texttt{/splatoon}): 全ブキを一覧表示する.
    \item \textbf{詳細表示} (GET \texttt{/splatoon/:index}): 指定ブキの詳細を表示する.
    \item \textbf{新規登録フォーム} (GET \texttt{/splatoon/create}): 登録画面へ遷移する.
    \item \textbf{新規登録処理} (POST \texttt{/splatoon}): データを追加する.
    \item \textbf{編集フォーム} (GET \texttt{/splatoon/edit/:index}): 編集画面を表示する.
    \item \textbf{更新処理} (POST \texttt{/splatoon/update/:index}): データを更新する.
    \item \textbf{削除処理} (GET \texttt{/splatoon/delete/:index}): データを削除する.
\end{itemize}

\end{document}